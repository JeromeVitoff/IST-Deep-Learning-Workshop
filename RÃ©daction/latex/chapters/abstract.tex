% !TEX root = manuscript.tex
\section*{Résumé}

Ce rapport présente les travaux réalisés dans le cadre du challenge Kaggle "IST Deep Learning Workshop" portant sur la reconnaissance de caractères ourdou. Développé en l’absence d’une alternance au second semestre de ma première année de Master MIASHS, ce projet s'inscrit dans un contexte académique visant à approfondir mes compétences en intelligence artificielle et en traitement d'images.

L'objectif est de concevoir un modèle d'apprentissage automatique capable de classifier correctement des images de caractères ourdou. Les données utilisées comprennent un ensemble d'images de dimensions 28×28 pixels, représentant 40 classes distinctes de caractères. Le projet s'articule autour d'étapes clés : prétraitement des données d'image, mise en œuvre progressive de modèles de complexité croissante (MLP, CNN simple, CNN profond avec BatchNormalization) et analyse comparative des performances.

Les résultats démontrent une amélioration significative des performances, passant d'une précision de 55-59\% avec le MLP à 88\% avec un CNN simple, pour atteindre 98,5\% avec le CNN profond. La soumission finale au challenge a obtenu un score remarquable de 99,1\%, validant l'efficacité de l'approche adoptée.

Au-delà de la résolution d'un problème de classification d'images, ce projet représente une opportunité unique d'explorer les techniques modernes d'apprentissage profond, tout en renforçant ma capacité à concevoir des solutions adaptées à des problématiques complexes de vision par ordinateur.

\section*{Abstract}

This report presents the work carried out as part of the Kaggle challenge "IST Deep Learning Workshop" on Urdu character recognition. This project was undertaken in an academic context aimed at deepening my skills in artificial intelligence and image processing.

The objective is to design a machine learning model capable of correctly classifying images of Urdu characters. The data used includes a set of 28×28 pixel images representing 40 distinct character classes. The project revolves around key steps: preprocessing of image data, progressive implementation of models of increasing complexity (MLP, simple CNN, deep CNN with BatchNormalization), and comparative performance analysis.

The results demonstrate a significant improvement in performance, from an accuracy of 55-59\% with the MLP to 88\% with a simple CNN, reaching 98.5\% with the deep CNN. The final submission to the challenge achieved a remarkable score of 99.1\%, validating the effectiveness of the adopted approach.

Beyond solving an image classification problem, this project represents a unique opportunity to explore modern deep learning techniques, while strengthening my ability to design solutions adapted to complex computer vision challenges.
\newpage