% !TEX root = manuscript.tex
\clearpage
\chapter*{Conclusion}
\vspace{-1cm}

\begin{flushleft}
Ce projet de reconnaissance de caractères ourdou a permis d'explorer une progression méthodique d'architectures de complexité croissante, démontrant clairement l'importance de l'adéquation entre la nature des données et la structure des modèles employés.

\bigskip

Nous avons observé une amélioration spectaculaire des performances à travers les trois principales approches explorées :
Le MLP simple a atteint une précision limitée de 54.6\%, illustrant les difficultés inhérentes au traitement d'images sans préserver leur structure spatiale. Le CNN simple a représenté une avancée significative avec 88.4\% de précision, confirmant l'importance des opérations de convolution pour l'analyse d'images. Le CNN profond a culminé à 98.5\% sur l'ensemble de validation et 99.1\% sur l'ensemble de test, démontrant l'efficacité des architectures profondes combinées à des techniques de régularisation modernes comme BatchNormalization.\newline
Cette progression de près de 45 points de pourcentage entre l'approche initiale et finale constitue un exemple frappant des gains de performance que peuvent apporter des choix architecturaux adaptés en apprentissage profond.\newline
Le développement progressif de modèles de complexité croissante s'est révélé être une stratégie pédagogiquement enrichissante, permettant de comprendre précisément comment chaque élément architectural contribue à l'amélioration des performances. L'analyse des matrices de confusion a également fourni des perspectives intéressantes sur les erreurs de classification, passant d'une confusion généralisée avec le MLP à des confusions très localisées et limitées avec le CNN profond. \newline

Sur le plan professionnel, cette expérience a permis d'acquérir des compétences directement transférables au monde industriel telles que: la capacité à traiter des problèmes de vision par ordinateur avec diverses architectures de deep learning, expérience dans l'optimisation des performances de modèles complexes, et la connaissance pratique des défis spécifiques liés à la reconnaissance de caractères non latins.\newline
 
Sur le plan personnel, ce projet a représenté une opportunité unique de développer des compétences techniques et méthodologiques essentielles comme: la Maîtrise progressive des architectures de deep learning, Analyse critique des résultats, Gestion d'un projet complet.

\bigskip
La réussite de ce projet, attestée par le score de 99.1\% sur la compétition Kaggle, constitue une validation objective de ces compétences et une base solide pour de futurs défis dans le domaine de l'intelligence artificielle appliquée au traitement d'images.

\end{flushleft}