% !TEX root = manuscript.tex
\begin{minipage}{.4\textwidth}
\begin{center}
\includegraphics[height=3cm]{logos/master.pdf}
\end{center}
\end{minipage}% This must go next to `\end{minipage}`
\hspace{2cm}
\begin{minipage}{.4\textwidth}
\begin{center}
\includegraphics[height=3cm]{logos/upv.eps}
\end{center}
\end{minipage}%


\begin{center}
\vspace{1cm}
{\huge UFR 6}\\
Université Paul Valéry, Montpellier III\\
\vspace{1cm}

{\Large Mémoire Professionnel S2M1}
\end{center}
\noindent\makebox[\linewidth]{\rule{\paperwidth}{0.4pt}}
\vspace{-15em}
{\let\newpage\relax\maketitle}
\vspace{-10em}
\noindent\makebox[\linewidth]{\rule{\paperwidth}{0.4pt}}

\begin{minipage}{.45\textwidth}
\vspace{3em}
Dans ce rapport, je retrace les étapes de
mes travaux sur le projet Kaggle intitulé
"IST Deep Learning Workshop:
Urdu Character recognition dataset for MLP". Ce projet a été conçu dans un contexte
de recherche et d’apprentissage, visant
à combler l’absence d’une alternance en
durant le second semestre de l’année universitaire.

\end{minipage}% This must go next to `\end{minipage}`
\hspace{0.5cm}
\begin{minipage}{.45\textwidth}
\vspace{3em}
Cette expérience m’a permis d’explorer des méthodologies modernes de machine learning, notamment l’application des techniques de Perceptron multicouche (MLP) et de  l'Apprentissage profond (Deep Learning)
dans le but d'identifier correctement des caractères ourdou isolés à partir d'images. J’ai également travaillé sur l’analyse des résultats pour en tirer des enseignements
concrets.  
\end{minipage}
\newpage